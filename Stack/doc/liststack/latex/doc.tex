\documentclass[11pt]{article} 
\usepackage[latin1]{inputenc} 
\usepackage[T1]{fontenc} 
\usepackage{textcomp}
\usepackage{fullpage} 
\usepackage{url} 
\usepackage{ocamldoc}
\begin{document}
\tableofcontents
\section{Module {\tt{Liststack}} : Last-in-first-out stack data structure.}
\label{module:Liststack}\index{Liststack@\verb`Liststack`}



 	This module implements stacks (LIFOs) using OCaml list



\ocamldocvspace{0.5cm}



\label{type:Liststack.t}\begin{ocamldoccode}
type {\textquotesingle}a t 
\end{ocamldoccode}
\index{t@\verb`t`}
\begin{ocamldocdescription}
The type of stacks containing the elements of type {\tt{{\textquotesingle}a}}.


\end{ocamldocdescription}




\label{exception:Liststack.Stack-underscoreempty}\begin{ocamldoccode}
exception Stack_empty
\end{ocamldoccode}
\index{Stack-underscoreempty@\verb`Stack_empty`}
\begin{ocamldocdescription}
Raised when {\tt{Liststack.pop}}[\ref{val:Liststack.pop}] or {\tt{Liststack.top}}[\ref{val:Liststack.top}] is applied to an empty stack.


\end{ocamldocdescription}




\label{val:Liststack.create}\begin{ocamldoccode}
val create : unit -> {\textquotesingle}a t
\end{ocamldoccode}
\index{create@\verb`create`}
\begin{ocamldocdescription}
Return a new stack, initially empty.


\end{ocamldocdescription}




\label{val:Liststack.is-underscoreempty}\begin{ocamldoccode}
val is_empty : {\textquotesingle}a t -> bool
\end{ocamldoccode}
\index{is-underscoreempty@\verb`is_empty`}
\begin{ocamldocdescription}
Return {\tt{true}} if the given stack is empty. Otherwise return {\tt{false}}.


\end{ocamldocdescription}




\label{val:Liststack.push}\begin{ocamldoccode}
val push : {\textquotesingle}a -> {\textquotesingle}a t -> {\textquotesingle}a t
\end{ocamldoccode}
\index{push@\verb`push`}
\begin{ocamldocdescription}
{\tt{push x s}} adds the element {\tt{x}} at the top of stack {\tt{s}} and returns the new stack.


\end{ocamldocdescription}




\label{val:Liststack.pop}\begin{ocamldoccode}
val pop : {\textquotesingle}a t -> {\textquotesingle}a t
\end{ocamldoccode}
\index{pop@\verb`pop`}
\begin{ocamldocdescription}
{\tt{pop s}} removes the top element of stack {\tt{s}} and return the new stack.


\end{ocamldocdescription}




\label{val:Liststack.top}\begin{ocamldoccode}
val top : {\textquotesingle}a t -> {\textquotesingle}a
\end{ocamldoccode}
\index{top@\verb`top`}
\begin{ocamldocdescription}
{\tt{top s}} returns the top element. Stack {\tt{s}} remains the same as before.


\end{ocamldocdescription}




\label{val:Liststack.print}\begin{ocamldoccode}
val print : {\textquotesingle}a t -> ({\textquotesingle}a -> string) -> unit
\end{ocamldoccode}
\index{print@\verb`print`}
\begin{ocamldocdescription}
Print the stack content to standard stdout.


\end{ocamldocdescription}


\end{document}